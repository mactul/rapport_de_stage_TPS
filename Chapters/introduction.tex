Le protocole TCP régit aujourd'hui l'immense majorité des flux sur Internet, en particulier tout le web jusqu'à HTTP/2 compris est construit sur TCP.
Malheureusement, TCP étant un protocole datant de 1974\up{\cite{tcp}}, il n'est plus parfaitement adapté à toutes nos applications, c'est pourquoi le protocole QUIC a été imaginé pour remplir les mêmes fonctions que TCP tout en étant beaucoup plus flexible.

QUIC est à la base d'HTTP/3, mais sa flexibilité lui permet aussi d'être utilisé pour des communications en conditions particulières, avec des taux de pertes importants ou des temps d'aller-retour très élevés; c'est pourquoi on retrouve QUIC utilisé pour certaines communications avec Mars.

Une des implémentations open-source de QUIC parmi les plus utilisées est Quiche, une bibliothèque en langage Rust conçue par Cloudflare (pour éviter toute ambiguïté, on référera cette implémentation par le nom cloudflare/quiche\up{\cite{cloudflare-quiche}}). Quiche se veut être fiable et sécurisé, mais ses performances sont très en dessous des performances de TCP.

Florentin Rochet a créé Quiceh (qu'on référera par le nom frochet/quiceh\up{\cite{frochet-quiceh}}) qui est une modification de Quiche censée améliorer ses performances. L'objectif de ce stage était d'intégrer frochet/quiceh à cURL, un client HTTP/3 et d'étudier les performances. La découverte d'un point bloquant dans les performances nous a amené à modifier directement cloudflare/quiche pour rajouter une nouvelle fonctionnalité, les accusés de réception différés.

\vspace{0.5cm}

Dans une première partie, nous présenterons l'Université de Namur, ses recherches et ses liens avec Cloudflare. Dans une seconde partie nous présenterons le travail effectué de façon chronologique, ce qui nous amènera à présenter les 3 gros projets sur lequel nous avons travaillés, Quiche, Quiceh et cURL.